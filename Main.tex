\documentclass[12pt,utf8]{beamer}

% Gute Einführung zu LaTeX-Beamer: http://www2.informatik.hu-berlin.de/~mischulz/beamer.html

%-----PARAMETERS-----

%Wichtige Standard Pakete!
\usepackage[ngerman]{babel}
\usepackage{xcolor}
\usepackage{graphicx}
\usepackage{tikz}

%Für den Header notwendig!
\usepackage[percent]{overpic}

%Für mehrere Spalten
\usepackage{multicol}

%Einbinden des Themes
\input{beamerthemeTU.sty}

%Standard Angaben
\title{Linux Vortrag}
\subtitle{Warum Linux cool ist...}

\author[J.-M. Lenk, C. Parnitzke, J. Schneider]{Jan-Marius Lenk, Christoph Parnitzke, Josef Schneider}
\institute[FOSS AG - FbI]{Free and Open Source Software AG\\ Fachbereich Informatik}

\date{\today}

%-----IMPLEMENTATION-----

\begin{document}

\begin{frame}
	\titlepage
\end{frame}

\section{Einleitung}

\begin{frame}{Inhaltsverzeichnis}
	\tableofcontents[currentsection, hideallsubsections]
\end{frame}

\begin{frame}{Wer sind die...}{...und was wollen die?}
	Die \textbf{F}ree and \textbf{O}pen \textbf{S}ource \textbf{S}oftware \textbf{AG} stellt sich vor... 
\end{frame}

\begin{frame}
\frametitle{Was machen wir?}
\begin{itemize}
	\item Vortragsreihen
	\item Installationspartys
	\item Orientierungsphase Informatik TU Dortmund
	\item Zusammenarbeit mit den lokalen Hackspaces
	\item Raspberry Pi Projekte an Schulen
	\item Linux-HelpDesk
\end{itemize}
\end{frame}

\begin{frame}
	\frametitle{Wo findet man uns?}
	\begin{itemize}
		\item foss-ag.de
		\item Social Media: Telegram, Riot, Twitter
		\item wöchentliche Treffen an der TU Dortmund\\(siehe Homepage)
	\end{itemize}
\end{frame}

\begin{frame}
\frametitle{Wir haben für euch geplant:}
\begin{figure}
\includegraphics[scale=0.1]{resources/linuxcalm.png}
\end{figure}
	\begin{itemize}
		\item Software \& Desktops unter Linux + offene Diskussionsrunde
		\item Linux - Installparty
	\end{itemize}
\end{frame}

\begin{frame}
\frametitle{Wir haben für euch geplant:}
	\begin{itemize}
		\item Dateisystem
		\item Einführung in die Kommandozeile
		\item Free my Android
		\item Python-Scripting
	\end{itemize}
\end{frame}

\begin{frame}
\frametitle{Wir haben für euch geplant:}
\begin{itemize}
	\item Advanced-Vortrag Kommandozeile
	\item SSH - Tipps und Tricks
	\item Backups mit Borg
	\item Bash, Fish und Co.
\end{itemize}
\end{frame}

%\begin{frame}[<-+>]{Wer sind die...}{...und was wollen die?}
%	\textquotedblleft Unser Ziel ist es freie, quelloffene oder gemeinnützige Projekte technisch zu unterstützen.\textquotedblright \\
%	Wir unterstützen ausserdem auch die Menschen selbst. Zum Beispiel euch! \\
%	Aber wie können wir euch am Besten helfen? Indem wir unser Wissen mit euch teilen. \\
%	Daher wollen wir euch jetzt unser Verständnis von FOSS und die Liebe zu Linux und freier Software euch nahe bringen.
%\end{frame}

%//end of section "Einleitung"

\section{Theorie}

\begin{frame}{Inhaltsverzeichnis}
	\tableofcontents[currentsection, hideallsubsections]
\end{frame}

\begin{frame}
	\begin{figure}
		\includegraphics[scale=1]{resources/open_swiss_knife.png}
	\end{figure}
\end{frame}

\begin{frame}
\centering\includegraphics[scale=0.5]{resources/hello_world_code.png}
\end{frame}

\begin{frame}
\centering\includegraphics[scale=0.3]{resources/hello_world_bin.png}
\end{frame}

\begin{frame}
	\begin{figure}
		\includegraphics[scale=0.4]{resources/pcwahl10.png}
	\end{figure}
\end{frame}

\begin{frame}
	\begin{figure}
		\includegraphics[scale=0.45]{resources/pmpc.png}
	\end{figure}
\end{frame}

%\\end of section "Theorie"

\section{Praxis}

\begin{frame}{Inhaltsverzeichnis}
	\tableofcontents[currentsection, hideallsubsections]
\end{frame}

\subsection{Distros}
\begin{frame}{Distributionen}
\note{Mint, OpenSuse, ubuntu, Fedora\\}
\note{alle direkt benutzbar}
\begin{columns}
\begin{column}{0.5\textwidth}
	\begin{figure}
		\includegraphics[width=0.9\textwidth]{resources/640px-Linux_Mint_logo_and_wordmark}
		%https://de.wikipedia.org/wiki/Linux_Mint#/media/File:Linux_Mint_logo_and_wordmark.svg
	\end{figure}
	
	\begin{figure}
		\includegraphics[width=0.8\textwidth]{resources/640px-OpenSUSE_Logo}
		%https://de.wikipedia.org/wiki/OpenSUSE#/media/File:OpenSUSE_Logo.svg
	\end{figure}
	
\end{column}
\begin{column}{0.5\textwidth}
	
	\begin{figure}
		\includegraphics[width=0.9\textwidth]{resources/640px-Ubuntu_logo}
		%https://de.wikipedia.org/wiki/Ubuntu#/media/File:Ubuntu_logo.svg
	\end{figure}
	
	\begin{figure}
		\includegraphics[width=0.9\textwidth]{resources/640px-Fedora_logo_and_wordmark}
		%https://de.wikipedia.org/wiki/Fedora_%28Linux-Distribution%29#/media/File:Fedora_logo_and_wordmark.svg
	\end{figure}
	
\end{column}
\end{columns}
\end{frame}

\subsection{Oberflächen}
\begin{frame}{Oberflächen}{Gnome}
\note{komplett tauschbare Oberflächen/DEs}
\note{Gnome3\\}
\begin{figure}
\includegraphics[height=0.6\textheight]{resources/1200px-GNOME_Shell.png}
%https://de.wikipedia.org/wiki/Gnome-Shell#/media/File:GNOME_Shell.png
\end{figure}

\end{frame}

\begin{frame}{Oberflächen}{KDE}

\note{KDE/Plasma\\}
\begin{figure}
\includegraphics[height=0.6\textheight]{resources/1200px-Kscreen-krunner.png}
%https://en.wikipedia.org/wiki/KDE_Plasma_5#/media/File:Kscreen-krunner.png
\end{figure}


\end{frame}

\begin{frame}{Oberflächen}{Cinnamon}
\note{Cinnamon/Mint\\}
\begin{figure}
\includegraphics[height=0.6\textheight]{resources/1200px-Linux_Mint.png}
%https://de.wikipedia.org/wiki/Cinnamon_(Desktop-Umgebung)#/media/File:Linux_Mint_13_RC.png
\end{figure}


\end{frame}

\subsection{Software}

\begin{frame}[allowframebreaks]{Software}
 
\note{Für Informatiker ist Linux geeignet, weil deren Software gut auf Linux läuft\\}
\hspace{1cm}

\begin{description}[style=nextline]
 \item [\textcolor{FOSSAGgreen}{\Large Browser}] {\bf Firefox}, Chromium, Vivaldi, Opera, Tor
  \item [\textcolor{FOSSAGgreen}{\normalsize statt}] Edge, Explorer, Safari
 \item [\textcolor{FOSSAGgreen}{\Large Office}] {\bf LibreOffice}, Kile (\LaTeX), \TeX maker
  \item [\textcolor{FOSSAGgreen}{\normalsize statt}] MS Office (365)
 \item [\textcolor{FOSSAGgreen}{\Large Email}] {\bf Thunderbird}, Icedove, Evolution 
  \item [\textcolor{FOSSAGgreen}{\normalsize statt}] Outlook
\pagebreak 
 \item [\textcolor{FOSSAGgreen}{\Large IDEs}] {\bf Eclipse}, IntelliJ, NetBeans, Atom, VI(M)
  \item [\textcolor{FOSSAGgreen}{\normalsize statt}] Visual Studio
 \item [\textcolor{FOSSAGgreen}{\Large Medien}]{\bf VLC}, Audacity, Rythmbox, Totem
\item [\textcolor{FOSSAGgreen}{\normalsize statt}] \textit{altem} Windows Media Player
 \item [\textcolor{FOSSAGgreen}{\Large Grafik}] {\bf GIMP}, Blender,  Inkscape
  \item [\textcolor{FOSSAGgreen}{\normalsize statt}] Photoshop, Illustrator, etc.
 \item [\textcolor{FOSSAGgreen}{\Large alles}] DAS TERMINAL 
\end{description}
 \end{frame}


 \begin{frame}{Software}{Wie bekomme ich die?}

  \begin{columns}
  \begin{column}{0.3\textwidth}
 \begin{figure}
 \includegraphics[height=0.5\textheight]{resources/garbage-296550_1280.png}
  %https://pixabay.com/en/garbage-electronics-trash-rubbish-296550/
 \end{figure}
\end{column}
\begin{column}{0.6\textwidth}
 \begin{center}
   \textbf{\large{ Paketverwaltung}}
 \end{center}
  \begin{itemize}
   \item Einfache Installation \note{globle liste der Software\\}
   \item Sicher \note{Auch hier muss man natürlich vertrauen, aber bei Linux muss man eben seltener ausweichen\\}
   \item Kein Balast \note{damit euer Rechner nicht im Müll versinkt\\}
  \end{itemize}
  \end{column}
  \end{columns}
 \end{frame}

\begin{frame}
	\frametitle{Zusammenfassung}
	\begin{itemize}
		\item Open-Source als Philosophie
		\item In vielen bereits Bereichen Standard
		\item Linux als freies Betriebssystem weit verbreitet
	\end{itemize}
\end{frame}

 






 







%\\end of section "Praxis"

\section{Schluß}

\begin{frame}{Das Beste}{zum Schluss:}
	Wir planen eine Party! Und zwar nur für euch! :)\\
	Eine Linux-Install-Party nur für euch:\\ \note{Was ist eine Installparty?\\}
	\begin{center}
	\begin{description}[<+->]
		\item[Wann?] Freitag: 14. Oktober 2016 Start: 15:00
		\item[Wo?] OH12 E003
		\item[Was?] Was ihr wollt, jedes System, jede Distro.
		\item[Womit?] Strom, Stick, Spaß    \note{Strom für Laptops, Sticks für Distro\\}
		\item[Koscht?] Nüscht
	\end{description}
	\end{center}
\begin{center}
	{\onslide<5-> \texttt{www.foss-ag.de}}
\end{center}

	\end{frame}

\begin{frame}{Image-sources - 1}
Dies sind die Quellen der Bilder, die für die präsentation benutzt wurden, sortiert nach ihrem Auftreten.
\begin{itemize}
	\item [1] 2.bp.blogspot.com/\textunderscore UqUwVPikChs/TFq5scy4dVI/AAAAAAAAOiM/tDuYjZGTSgY/s1600/GrandmaLinux.jpg
	\item [2] upload.wikimedia.org/wikipedia/commons/a/af/Tux.png
	\item [3] www.extremetech.com/wp-content/uploads/2014/02/02DataFlowBills3-1390852937757.jpg
	
	\item [4] upload.wikimedia.org/wikipedia/commons/c/c7/121212\textunderscore 2\textunderscore OpenSwissKnife.png
	\item [5] www.bizcoder.com/Media/Bizcoder/Windows-Live-Writer/715c931eba8c\textunderscore 7522/AllTheThings\textunderscore 2.jpg
	
\end{itemize}

\end{frame}

\begin{frame}{Image-sources - 2}
\begin{itemize}
	\item [6] danlynch.org/wp-content/uploads/2009/04/funny-pictures-your-kitten-uses-linux.jpg
	\item [7] www.valiantsolutions.com/images/infosec.jpg
	\item [8] https://pixabay.com/en/garbage-electronics-trash-rubbish-296550/
\end{itemize}
\end{frame}


%\\end of section "Schluß"

\begin{frame}
\vfill
\begin{center}\begin{Huge}Vielen Dank \\
für eure Aufmerksamkeit. \\[50pt]
**Free Discussion**\end{Huge}\vfill
\end{center}
\vfill
\end{frame}

\end{document}
