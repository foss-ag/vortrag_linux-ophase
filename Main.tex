\documentclass[12pt,utf8]{beamer}

% Gute Einführung zu LaTeX-Beamer: http://www2.informatik.hu-berlin.de/~mischulz/beamer.html

%-----PARAMETERS-----

%Wichtige Standard Pakete!
\usepackage[ngerman]{babel}
\usepackage{xcolor}
\usepackage{graphicx}
\usepackage{tikz}

%Für den Header notwendig!
\usepackage[percent]{overpic}

%Einbinden des Themes
\input{beamerthemeTU.sty}

%Standard Angaben
\title{Linux Vortrag}
\subtitle{Warum Linux cool ist...}

\author[J. Schneider]{Josef Schneider}
\author[J.-M. Lenk]{Jan-Marius Lenk}
\author[C. Parnitzke]{Christoph Parnitzke}
\institute[FOSS AG - FbI]{Free and Open Source Software AG\\ Fachbereich Informatik}

\date{\today}

%-----IMPLEMENTATION-----

\begin{document}

\begin{frame}
	\titlepage
\end{frame}

\begin{frame}[TU]{Inhaltsverzeichnis}
	\tableofcontents[hideallsubsections]
\end{frame}

\section{Motivation}

\begin{frame}{Inhaltsverzeichnis}
	\tableofcontents[currentsection, hideallsubsections]
\end{frame}

\input{Motivation.tex}

%//end of section "Motivation"

\section{Konzept}

\begin{frame}{Inhaltsverzeichnis}
	\tableofcontents[currentsection, hideallsubsections]
\end{frame}

\input{Konzept.tex}

%\\end of section "Konzept"

\section{Bramble}

\begin{frame}{Inhaltsverzeichnis}
	\tableofcontents[currentsection, hideallsubsections]
\end{frame}

\input{Bramble.tex}

%\\end of section "Bramble"

\section{Briar}

\input{Briar.tex}

%\\end of section "Briar"

\begin{frame}
\vfill
\begin{center}\begin{Huge}Vielen Dank \\
für eure Aufmerksamkeit. \\[50pt]
**Free Discussion**\end{Huge}\vfill
\end{center}
\vfill
\end{frame}

\end{document}
